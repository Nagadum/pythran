\documentclass[a4paper]{article}

\usepackage[utf8]{inputenc}   
\usepackage[T1]{fontenc}      
\usepackage[francais]{babel}  

\usepackage{url}
\usepackage{listings}
\lstset{language=python}

\usepackage{todonotes}
%\usepackage[disable]{todonotes}


\usepackage[a4paper]{geometry}

\title{Extension du langage Pythran pour le module itertools}           
\author{Alan Raynaud}
\date{}                      

\sloppy 

\begin{document}

%Page de garde

\maketitle   

\listoftodos[Liste de ce qui est tout doux]

\clearpage

\section*{Le langage pythran}

\todo{Description de pythran et de son utilité}

\section*{Objectifs}

\todo{Cout de la mémoire en terme de perf, objectif = zero inutile}

\section*{Le module itertools}

\todo{Implementation d'une lib standart de python donnant des
  operateurs permettant de ne pas gaspiller}

\section*{L'optimisation de code}

\todo{Utilisation des operateurs pour remplacer : genexp en iterateur,
  list comp, et autre inutiles}

\section*{Résultats obtenus}

\todo{Mesures de gain = super! Par contre, pas toujours possible}

\section*{Conclusion}

\todo{Suprises dues aux subtilités de la synthaxe python. Perspective
  = augmenter le taux de remplacement.}

\end{document}

